\documentclass[12pt]{article}
\usepackage[MeX]{polski}
\usepackage[utf8]{inputenc}
\usepackage{graphicx}
\usepackage{amsmath} %pakiet matematyczny
\usepackage{amssymb} %pakiet dodatkowych symbol
\usepackage{hyperref}
\begin{document}

Zasady
\begin{enumerate}
\item egzamin odbędzie się z wykorzystaniem platformy MS Teams na dedykowanym zespole, do którego należy dołączyć za pomocą kodu \textbf{7fwwx2u}, dołączenie powinno nastąpić na co najmniej dwa dni przed egzaminem
\item student przed przystąpieniem do egzaminu powinien posiadać uzupełnione i aktualne zdjęcie w systemie USOS (zakładka Dla Wszystkich - zdjęcie do legitymacji)
\item dla każdego studenta przystępującego do egzaminu zostanie utworzony oddzielny prywatny kanał, na którym będzie odbywał się jego egzamin
\item ok godziny 10:00-10:10 (lub innej ustalonej porze) na prywatnym kanale zostaną umieszczone polecenia do rozwiązania, przed przystąpieniem do ich rozwiązania student powinien rozpocząć spotkanie online w danym kanale i przebywać na nim do czasu przesłania rozwiązań, w trakcie spotkania należy mieć włączoną kamerę i udostępnianie ekranu
\item rozwiązania powinny być zamieszczone na prywatnym kanale w czasie przeznaczonym na egzamin, rozwiązanie przesłane po terminie mogą nie być sprawdzane
\item rozwiązania powinny być przesłane jako zdjęcia lub skany odręcznych rozwiązań
\item w trakcie egzaminu zabroniona jest komunikacja z innymi osobami w celu uzyskania nieuprawnionej pomocy
\item przebieg egzaminy może być utrwalany
\item egzaminatorzy okresowo będą dołączać do spotkań i weryfikować samodzielność rozwiązywania poleceń, w przypadku naruszeń - ocena niedostateczna w danym terminie
\item student najpóźniej na dzień przed egzaminem powinien zweryfikować poprawność działania własnego sprzętu (kamera/mikrofon, opcja udostępniania ekranu) - w razie problemów można skontaktować się z dr Piotrem Jastrzębskim piotr.jastrzebski@uwm.edu.pl lub poprzez wiadomość prywatną na MS Teams
\end{enumerate}
\end{document}