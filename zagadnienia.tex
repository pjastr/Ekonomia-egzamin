\documentclass[12pt]{article}
\usepackage[MeX]{polski}
\usepackage[utf8]{inputenc}
\usepackage{graphicx}
\usepackage{amsmath} %pakiet matematyczny
\usepackage{amssymb} %pakiet dodatkowych symbol
\usepackage{hyperref}
\usepackage{ulem}
\begin{document}

Zagadnienia na egzamin dyplomowy - wersja 2:
\begin{enumerate}
\item operacje arytmetyczne na macierzach
\item wyznacznik macierzy
\item macierz odwrotna
\item układy równań - wzory Cramera
\item \sout{rząd macierzy}
\item układ równań - eliminacja Gaussa
\item granice ciągu (wyrażenia wymierne, trygonometryczne, potęgowe, wykładnicze, "sprowadzalne do znanych granic")
\item granice funkcji (wyrażenia wymierne, trygonometryczne, potęgowe, wykładnicze, "sprowadzalne do znanych granic", reguła de l'Hospitala)
\item pochodna funkcji (reguły różniczkowania: pochodna sumy, różnicy, iloczynu i ilorazu; pochodna funkcji złożonej)
\item zastosowania pochodnej (równanie stycznej, wyznaczanie ekstremów, wyznaczanie przedziałów monotoniczności)
\item całka nieoznaczona (reguły całkowania: całka sumy i różnicy, całkowanie przez podstawienie i przez części, całki z wykorzystaniem wzorów trygonometrycznych, całki funkcji wymiernych)
\item \sout{całka oznaczona (wzór Leibniza-Newtona)}
\end{enumerate}
\end{document}