\documentclass[12pt,a4paper]{report}

\usepackage{amsmath}

\usepackage[utf8]{inputenc}

\usepackage{enumerate}

\usepackage[pdftex]{graphicx}\usepackage[MeX]{polski}

\textwidth = 480pt

\textheight = 660pt

\addtolength{\voffset}{-2.5cm}

\addtolength{\hoffset}{-2.5cm}

\begin{document}

\begin{center}

\textbf{MATEMATYKA \\12 LUTEGO 2021 - EGZAMIN}

\end{center}\textbf{Zadanie 1.} Wyznacz macierz odwrotną do macierzy \\\\\ $\mathbf{A} =\left[ \begin{array}{ccc}2 & 1& 0\\2& 0& 1\\1 & -1& 1\end{array} \right]$\\\\\\\textbf{Zadanie 2.} \\Rozwiąż wzorami Cramera. $\left\{ \begin{array}{ll}x+2y-3z=-3\\ 3x+y+4=2\\2x+3y+z=-2\end{array} \right.$\\\\\\\textbf{Zadanie 3.} \\Rozwiąż metodą eliminacji Gaussa. $\left\{ \begin{array}{ll}2x+2y-3z=-3\\3x+y=2\\x+2y+z=0\end{array} \right.$\\\\\\\textbf{Zadanie 4.} \\Oblicz granice funkcji.\\a)$\lim\limits_{x\to 0}\frac{1-\cos(3x)}{\sin^2(4x)}$\ \ \ \ \ \  b)$\lim\limits_{x\to +\infty}\frac{x^2+3}{x+1}$\ \ \ \ \ \ c)$\lim\limits_{x\to +\infty}\left( 1+\frac{1}{x-3}\right)^{2x+3}$\\\\\\\textbf{Zadanie 5.} \\ Oblicz pochodną funkcji $f(x)=\frac{x\cdot 2^x}{x^2+3x}$ w punkcie $x_0=1$.\\\\\\\textbf{Zadanie 6.} \\Oblicz całki. \\a)$\int x\sqrt{3-x^2} dx$ \ \ \ \ \ \ \ \ b)$\int \frac{3x}{x^2+4} dx$\\\\Punktacja: Każde zadanie numerowane jest po 6 pkt.



Widełki ocen:

\begin{itemize}

\item poniżej 18 pkt -- 2 (ndst)

\item od 18 pkt -- 3 (dst)

\item od 22 pkt -- 3,5 (dst+)

\item od 26 pkt -- 4 (db)

\item od 30 pkt -- 4,5 (db+)

\item od 33 pkt -- 5 (bdb)

\end{itemize}

\end{document}
