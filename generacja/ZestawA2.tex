\documentclass[12pt,a4paper]{report}

\usepackage{amsmath}

\usepackage[utf8]{inputenc}

\usepackage{enumerate}

\usepackage[pdftex]{graphicx}\usepackage[MeX]{polski}

\textwidth = 480pt

\textheight = 660pt

\addtolength{\voffset}{-2.5cm}

\addtolength{\hoffset}{-2.5cm}

\begin{document}

\begin{center}

\textbf{MATEMATYKA \\5 LUTEGO 2021 - EGZAMIN - PIERWSZY TERMIN}

\end{center}\textbf{Zadanie 1.} Dla podanych poniżej macierzy $A$ i $B$ wyznaczyć wyznacznik macierzy $A$ oraz iloczyn $A\cdot B$. \\\\ $\mathbf{A} =\left( \begin{array}{cccc}4 & 2& 5 & 6\\1 & -3& 8 & -9\\0 & -2& 1 & -1\\2 & 3 & -8 & 9 \end{array} \right)$ ,\ \ $\mathbf{B} =\left( \begin{array}{cccc}2 & -1\\1 & -5\\0 & 3\\4 & 0  \end{array} \right)$\\\\\\\textbf{Zadanie 2.} \\Rozwiąż wzorami Cramera. $\left\{ \begin{array}{ll}5x-3y-z=0\\-3x-y+2z=0\\x+y+z=8\end{array} \right.$\\\\\\\textbf{Zadanie 3.} \\Rozwiąż metodą eliminacji Gaussa. $\left\{ \begin{array}{ll}8x-12y+16z=-28\\-2x+3y-4z=7\\10x-15y+20z=-35\end{array} \right.$\\\\\\\textbf{Zadanie 4.}\\Oblicz granice ciągów.\\a)$\lim\limits_{n\to\infty}\frac{n^4+n^3}{7n+5n^3+8n^4}$\ \ \ \ \ \  b)$\lim\limits_{n\to \infty}(\frac{2n+5}{2n})^n$\ \ \ \ \ \ c)$\lim\limits_{n\to\infty} \frac{\sqrt{5n+9n^2}}{\sqrt[3]{8n^3+6n}}$\\\\\\\textbf{Zadanie 5.} \\Wyznacz równanie stycznej do wykresu funkcji $f(x)=\frac{x\cdot e^x}{\sqrt{x^2+4}}$ w punkcie $(0,0)$.\\\\\\\textbf{Zadanie 6.} \\Oblicz całki. \\a)$\int x^3\cdot e^{x^2} dx$ \ \ \ \ \ \ \ \ b)$\int \frac{dx}{x^2-6x+5} dx$\\\\Punktacja: Każde zadanie numerowane jest po 6 pkt.



Widełki ocen:

\begin{itemize}

\item poniżej 18 pkt -- 2 (ndst)

\item od 18 pkt -- 3 (dst)

\item od 22 pkt -- 3,5 (dst+)

\item od 26 pkt -- 4 (db)

\item od 30 pkt -- 4,5 (db+)

\item od 33 pkt -- 5 (bdb)

\end{itemize}

\end{document}
