\documentclass{report}
\usepackage{amsmath}
\usepackage[utf8]{inputenc}
\usepackage{enumerate}
\usepackage[pdftex]{graphicx}
\textwidth = 480pt
\textheight = 660pt
\addtolength{\voffset}{-2.5cm}
\addtolength{\hoffset}{-2.5cm}
\begin{document}
\begin{center}
\textbf{MATEMATYKA \\LUTY 2021 - EGZAMIN}
\end{center}
\textbf{Zadanie 1.} 
Dla podanych poniżej macierzy $A$ i $B$ wyznaczyć wyznacznik macierzy $A$ oraz iloczyn $A\cdot B$.
\\\\
$
\mathbf{A} =
\left( \begin{array}{cccc}
5 & -1& -2 & 0\\
4 & -1&1 & 6\\
6 & -3& 2 & 2\\
-5 & 1 & 2 & 4 
\end{array} \right)
$ ,\ \ 
$
\mathbf{B} =
\left( \begin{array}{ccc}
2 & 0& 0\\
0 & 0& -5\\
0 & 3& 0\\
4 & 0& 0
\end{array} \right)
$
\\\\\\\textbf{Zadanie 2.} 
\\Rozwi\c{a}\.z metod\c{a} Cramera. 
$
\left\{ \begin{array}{ll}
4x+3y-2z=0\\
-x-y+z=2\\
8y-z=11
\end{array} \right.
$
\\\\\\\textbf{Zadanie 3.} 
\\Rozwi\c{a}\.z metod\c{a} Gaussa. 
$
\left\{ \begin{array}{ll}
3x+5y+2z=10\\
-x+3z=2\\
4x+5y-z=8
\end{array} \right.
$
\\\\\\\textbf{Zadanie 4.}
\\Oblicz granice funkcji.
\\a)$\lim\limits_{x\to 3}\frac{x^2-9}{\vert x-3\vert}$
\ \ \ \ \ \  b)$\lim\limits_{x\to 7}\frac{x^2+3}{x+1}$
\ \ \ \ \ \ c)$\lim\limits_{x\to 0} \frac{x+sinx}{e^x-1}$
\\\\\\\textbf{Zadanie 5.}
\\Wyznacz przedziały monotoniczności oraz ekstrema lokalne funkcji $f(x)=\frac{3x+2}{x^2-4x+5}$.
\\\\\\\textbf{Zadanie 6.}
\\Oblicz całki. 
\\a)$\int x^2\cdot e^{-x} dx$ \ \ \ \ \ \ \ \ b)$\int \frac{dx}{x^2-7x+10} dx$
\end{document}
