\documentclass{report}
\usepackage{amsmath}
\usepackage[utf8]{inputenc}
\usepackage{enumerate}
\usepackage[pdftex]{graphicx}
\textwidth = 480pt
\textheight = 660pt
\addtolength{\voffset}{-2.5cm}
\addtolength{\hoffset}{-2.5cm}
\begin{document}
\begin{center}
\textbf{MATEMATYKA \\LUTY 2021 - EGZAMIN}
\end{center}
\textbf{Zadanie 1.} 
Dla podanych poniżej macierzy $A$ i $B$ wyznaczyć wyznacznik macierzy $A$ oraz iloczyn $A\cdot B$.
\\\\
$
\mathbf{A} =
\left( \begin{array}{cccc}
4 & 6& -5 & 10\\
5 & -2& 1 & -3\\
0 & -2& 1 & -1\\
2 & 3 & -4 & 5 
\end{array} \right)
$ ,\ \ 
$
\mathbf{B} =
\left( \begin{array}{cccc}
2 & -1\\
1 & -5\\
0 & 3\\
4 & 0  
\end{array} \right)
$
\\\\\\\textbf{Zadanie 2.} 
\\Rozwi\c{a}\.z metod\c{a} Gaussa. 
$
\left\{ \begin{array}{ll}
8x-12y+16z=-28\\
-2x+3y-4z=7\\
10x-15y+20z=-35
\end{array} \right.
$
\\\\\\\textbf{Zadanie 3.} 
\\Rozwi\c{a}\.z metod\c{a} Cramera. 
$
\left\{ \begin{array}{ll}
x+y+2z=-3\\
4x+2y+3z=5\\
5y+4z=0
\end{array} \right.
$
\\\\\\\textbf{Zadanie 4.}
\\Oblicz granice ci\c{a}g\'ow.
\\a)$\lim\limits_{n\to\infty}\frac{n^4+n^3}{7n+5n^3+8n^4}$
\ \ \ \ \ \  b)$\lim\limits_{n\to \infty}(\frac{2n+5}{2n})^n$
\ \ \ \ \ \ c)$\lim\limits_{n\to\infty} \frac{\sqrt{5n+9n^2}}{\sqrt[3]{8n^3+6n}}$
\\\\\\\textbf{Zadanie 5.}
\\Wyznacz równanie stycznej do wykresu funkcji $f(x)=\frac{xe^{x^2}}{\sqrt{x+9}}$ w punkcie $(0,0)$. 
\\\\\\\textbf{Zadanie 6.}
\\Oblicz całki. 
\\a)$\int \operatorname{arctg}(4x) dx$ \ \ \ \ \ \ \ \ b)$\int \frac{dx}{x^2-x-20} dx$
\end{document}
